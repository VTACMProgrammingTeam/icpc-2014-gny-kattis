\problemname{Height Ordering}

Mrs. Chambers always has her class line up in height order (shortest
at the front of the line).  Every September a new class of exactly 20
3rd graders arrive, all of different height. For the first few days it
takes a long time to get the kids in height order, since no one knows
where they should be in the line.  Needless to say, there is quite a
bit of jockeying around.  This year Mrs. Chambers decided to try a new
method to minimize this ordering chaos.  One student would be selected
to be the first person in line.  Then, another student is selected and
would find the \emph{first} person in the line that is taller than him, and
stand in front of that person, thereby causing all the students behind
him to step back to make room.  If there is no student that is taller,
then he would stand at the end of the line.  This process continues,
one student at-a-time, until all the students are in line, at which
point the students will be lined up in height order.

For this problem, you will write a program that calculates the total
number of steps taken back during the ordering process for a given class
of students.

\section*{Input}

The first line of input contains a single integer $P$, ($1 \le P \le 1000$) 
which is the number of data sets that follow.  Each data set should be 
processed identically and independently.

Each data set consists of a single line of input.  It contains the data
set number, $K$, followed by $20$ non-negative unique integers separated by
a single space.  The 20 integers represent the height (in millimeters)
of each student in the class.

\section*{Output}

For each data set there is one line of output.  The single output line
consists of the data set number, $K$, followed by a single space followed
by total number of steps taken back.

