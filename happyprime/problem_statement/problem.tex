\problemname{Happy Happy Prime Prime}

\noindent
\textbf{RILEY VASHTEE}: [\textit{reading from display}]  Find the next number in the sequence:

\begin{center}
    \textbf{313 331 367 ...? What?}
\end{center}

\noindent
\textbf{THE DOCTOR: 379.}

\noindent
\textbf{MARTHA JONES:} What?

\noindent
\textbf{THE DOCTOR:} It’s a sequence of happy primes – \textbf{379}.

\noindent
\textbf{MARTHA JONES:} Happy \textit{what?}

\noindent
\textbf{THE DOCTOR}: Any number that reduces to one when you take the sum of the
square of its digits and continue iterating it until it yields $1$ is a
happy number. Any number that doesn't, isn't.  A \emph{happy} prime is both
happy and prime.

\noindent
\textbf{THE DOCTOR}: I dunno, talk about \emph{dumbing down}.  
Don't they teach recreational mathematics anymore?

\noindent
Excerpted from ``\textit{Dr. Who},'' Episode 42 (2007).

The number 7 is certainly prime.  But is it happy?

\begin{eqnarray*}
    7 & \rightarrow & 7^2 = 49 \\
    49 & \rightarrow & 4^2 + 9^2 = 97 \\
    97 & \rightarrow & 9^2 + 7^2 = 130 \\
    130 & \rightarrow & 1^2 + 3^2 = 10 \\
    10 & \rightarrow & 1^2 + 0^2 = 1 \\
\end{eqnarray*}

\noindent
It is happy \texttt{:-)}
As it happens, 7 is the smallest happy prime.  
Please note that for the purposes of this problem, 1 is \emph{not} prime.

\noindent
For this problem you will write a program to determine if a number is a happy prime.

\section*{Input}

The first line of input contains a single integer $P$, ($1 \le P \le 10000$),
which is the number of data sets that follow.  Each data set should be
processed identically and independently.

Each data set consists of a single line of input.  It contains the data set number, $K$, 
followed by the happy prime candidate, $m$, ($1 \le m \le 10000$).

\section*{Output}

For each data set there is a single line of output.  The single output
line consists of the data set number, $K$, followed by a single space
followed by the candidate, $m$, followed by a single space, followed by
\texttt{YES} or \texttt{NO}, indicating whether $m$ is a happy prime.


